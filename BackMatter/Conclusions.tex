%===================================================================================
% Chapter: Conclusiones
%===================================================================================
\chapter*{Conclusiones}\label{chapter:conclusions}
\addcontentsline{toc}{chapter}{Conclusiones}

  
%===================================================================================
Tras un profundo estudio del estado del arte de la segregación residencial, en este trabajo se utilizaron dos metodologías, Kmeans y Trayectorias Simples, para realizar un estudio para determinar la existencia de la segregación en la ciudad de la Habana. En un primer punto se ha de destacar la creación de una base de datos que servirá de base para futuros estudios sobre el tema. Los datos son una compilación entre el Censo poblacional del 2012 provisto por la ONEI y ubicaciones geográficas dadas por GEOCUBA.

El análisis de la segregación estuvo centrado en el envejecimiento de las viviendas en la ciudad de la Habana. La selección no fue para nada aleatoria se evidenció en el trabajo lo preocupante que puede llegar a ser la situación y que existen pocos trabajos al respecto. En cuanto a las metodologías seleccionadas para Kmeans se realizó revisión bibliográfica de los métodos para seleccionar el número óptimo de clústeres iniciales, incluso se realizó una propuesta basada en el algoritmo de Trayectorias Simples.  Se propuso una variante para el algoritmo de Trayectorias Simples cambiando el orden de recorrido en una trayectoria, se cambió recorrer según el punto más cercano por recorrer según el valor de la variable de estudio más próximo.
 
Para validar las metodologías se realizaron comparaciones entre las mismas, tanto entre el algoritmo de Trayectorias Simples y Trayectorias Simples Modificadas como Trayectorias Simples Modificadas y Kmeas. Ambas comparaciones ofrecieron información determinante para la selección de una u otra metodología, sin embargo, cual usar en cada momento dependerá enteramente de los objetivos de la investigación en la que se vaya a aplicar y en la naturaleza del problema que se quiera estudiar. Un factor decisivo entre qué tipo de algoritmo de trayectoria utilizar será si es conveniente recorrer la ciudad por cercanía geográfica o por semejanza de comportamiento. Mientras que para decidir si usar trayectorias o Kmeans es necesario determinar si se quiere analizar el problema de una manera global o de una manera local.

Si bien se cumplieron los objetivos trazados, siempre queda trabajo por hacer respecto a esta metodología. Teniendo en cuenta lo anterior, se proponen algunas recomendaciones:
\begin{enumerate}
	\item Realizar un análisis de segregación a nivel e municipios utilizando Kmeans
	\item En las Trayectorias modificadas, estimar el intervalo de confianza más ajustado para cada caso
	\item Realizar otros estudios para detectar segregación en la Habana tomando otras variables: Nivel de Escolaridad, Ingresos
	\item Realizar un estudio similar a nivel de país
\end{enumerate}