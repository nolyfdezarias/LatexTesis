%===================================================================================
% Chapter: Introducción
%===================================================================================
\chapter*{Introducción}\label{chapter:introduction}
\addcontentsline{toc}{chapter}{Introducción}

La segregación residencial es un concepto que nos acompaña desde la década del 20 en el pasado siglo. En los últimos años los estudios sobre las condiciones de vida, las desigualdades, la diferenciación socio-espacial, la vulnerabilidad, la pobreza y las brechas sociales, han alcanzado una gran importancia en diversas disciplinas principalmente en las ciencias sociales. La compresión de dicho fenómeno tiene un gran significado para las instituciones públicas y gubernamentales. Las mismas, respaldadas en estos análisis, tienen oportunidad de mejorar las políticas y condiciones impulsando un acceso más justo e igualitario a los servicios e infraestructuras en cada localidad.

En la literatura existen dos definiciones fundamentales para la segregación residencial. La visión jerárquica \cite{Ortega2018LaSR}, definida por el proceso donde los grupos étnicos de mayor poder condicionan las oportunidades de acceso al suelo urbano a aquellos grupos de menos poder. Por otra parte, el enfoque geográfico \cite{Merkel2014QueEY}, en el que estará basado este estudio, está caracterizado por las relaciones espaciales, donde se afirma que cualquier grupo desigualmente distribuido en un espacio presenta segregación.

Las marcadas diferencias sociales determinan la composición de los vecindarios y afectan en gran medida muchos aspectos de la vida urbana: el acceso a educación, a la salud y a los servicios en general. Aquí radica la necesidad de los estudios de segregación que permitan encontrar vulnerabilidades en las ciudades con el fin de erradicarlas o mejorar los indicadores. 

En la literatura se reportan numerosos estudios sobre la segregación residencial en todas las ciencias sociales, principalmente en la sociología. Analizar dicho fenómeno no es una tarea sencilla pero debido al notable crecimiento de la ciencia de datos, análisis estadísticos de grandes volúmenes de información y el aprendizaje por computadora han provocado que estas tecnologías se viertan también en el estudio de segregación. Existen diferentes metodologías y métodos a la hora de abordar el problema en cuestión. Uno de las técnicas más novedosas es el uso de algoritmos de clustering \cite{Morissette2013TheKC}. La presente investigación se centra en el desarrollo de un algoritmo que permita detectar segregación residencial empleando los algoritmos de clustering y las Trayectorias Simples Modificadas \cite{RandonFurling2018FromUS}. Una trayectoria constituye una curva que recorre todo un territorio divido en pequeñas estructuras partiendo de un punto inicial. Normalmente en la literatura se hace referencia a las estructuras en la que queda dividida un territorio con el nombre de bloques censales. Las trayectorias simples son aquellas donde en cada bloque censal de la trayectoria se almacena una proporción entre la variable de estudio y la población total en dicho bloque.\\\\


Por lo anteriormente expuesto, se establecen los siguientes objetivos:


\subsection*{Objetivo general:}
Detectar la existencia de segregación mediante una propuesta de solución basada en algoritmos de clustering.
\subsection*{Objetivos específicos:}
\begin{itemize}
	
	\item Estudiar el estado del arte de los métodos de trayectorias y algoritmos de clustering utilizados en investigaciones sobre la segregación residencial.
	
	\item Definir métricas de comparación para evaluar la factibilidad de emplear dichos algoritmos en la detección de diferentes tipos de segregación residencial.
	

	
\end{itemize}



Este trabajo estará divido en 3 capítulos. En el capítulo 1 se resumen los principales resultados en materia de segregación y se discuten los algoritmos de clustering empleados en este tipo de problemas. En el capítulo 2 se formaliza el problema central que se aborda en este trabajo, se propone una estrategia para analizar la \textbf{segregación poblacional} y se formula una medida para su validación. En el capítulo 3 se describen los experimentos realizados. Finalmente aparecen las conclusiones, recomendaciones y la bibliografía consultada.

%\subsection*{Contribuciones}

%\subsection*{Organización de la Tesis}

%===================================================================================